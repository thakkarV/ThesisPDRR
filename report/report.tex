%%%%%%%%%%%%%%%%%%%%%%%%%%%%%%%%%%%%%%%%%
% University Assignment Title Page 
% LaTeX Template
% Version 1.0 (27/12/12)
%
% This template has been downloaded from:
% http://www.LaTeXTemplates.com
%
% Original author:
% WikiBooks (http://en.wikibooks.org/wiki/LaTeX/Title_Creation)
%
% License:
% CC BY-NC-SA 3.0 (http://creativecommons.org/licenses/by-nc-sa/3.0/)
%%%%%%%%%%%%%%%%%%%%%%%%%%%%%%%%%%%%%%%%%
\documentclass[12pt]{article}
\usepackage[english]{babel}
\usepackage[utf8x]{inputenc}
\usepackage{amsmath}
\usepackage{graphicx}
\usepackage[colorinlistoftodos]{todonotes}

\begin{document}
\begin{titlepage}
\begin{center}
\newcommand{\HRule}{\rule{\linewidth}{0.5mm}}
\textsc{\LARGE Boston University}
\linebreak
\textsc{\Large Electrical and Computer Engineering}\\[0.5cm]
\HRule \\[0.4cm]
{\huge \bfseries Senior Thesis PDRR Report}\\[0.4cm] % Title of your document
\HRule \\[1.5cm]
 
\begin{minipage}{0.4\textwidth}
\begin{flushleft} \large
\emph{Author:}\\
Vijay Thakkar
\end{flushleft}
\end{minipage}
~
\begin{minipage}{0.4\textwidth}
\begin{flushright} \large
\emph{Supervisor:}\\
Dr. Richard West
\end{flushright}
\end{minipage}\\[2cm]

{\large \today}\\[2cm] 

\includegraphics[scale=0.15]{./../img/buseal.png}\\[1cm]
\vfill
\end{center}
\end{titlepage}


\tableofcontents
\pagebreak

%%%%%%%%%%%%%%%%%%%%%%%%%%%%%%%%%%%%%%%%%%%%%%%%%%%%%%%%%%%%%%%%%%%%%%%%%%%%%%%%
\section{Thesis}
\begin{center}
\large Design and implement a proof of concept system for low latency and low overhead cross virtual machine communication between an Android OS and a Quest RTOS emulated on x86 multi-core processors while ensuring the timing constrains of the mixed criticality domain.
\end{center}

\section{Summary}
We intend to design and implement a proof of concept system for the future autonomous car platforms that allow for mixed criticality processing on a consolidated high performance system on chip. We have chosen x86 processors as the platform of choice due to client requirements as well as the ready availability, widespread use and proven functionality of multi-core x86 processors. We believe that time is a first class resource and therefore require that the final platform have a Real Time Operating System that is responsible for the high criticality control processing of the car. In addition, our client requires that we have an Android OS also running on the same platform to serve as the low criticality (info-tianment) processing. However, we still need to enable some cross virtual machine communication to maintian some shared data such as CAN bus messages of the car in a way that does not interfere with the timing guarantees of the RTOS control processing. The goal of the thesis is to design and implement a proof of concept system that allows such cross-VM communication within a mixed criticality domain.


\section{Motivation}
Traditionally, vehicle control systems have comprised of many dozens of microprocessors that control the car via relays and communicate with each other over a Controller Area Network (CAN) bus. These micro-controllers often have architectures that are not widely supported by mainstream software toolchains and are not powerful enough to run compute heavy software such as Android CarPlay systems. This poses many limitations on modern cars due to their heavy integration of automated navigation systems, infotainment and even autonomous driving.

The approach has been to include a ARM processor and/or an NVidia GPU in addition to the set of micro controllers specifically to serve these modern functions with the control network completely isolated. This results in a complex design and increased costs that can be eliminated if a single consolidated platform for both control systems as well as infotainment and a host for automated driving were to be used.

The approach taken by Professor Rich West and his team has been to design and build such a platform for automobiles of the future that use a single multi-core x86 processor as a platform for hosting all of the functionality without compromises. The intended design of this plat form has Quest-V as a partitioning hypervisor with an instance of Quest RTOS guest for time-critical control systems processing, a Linux guest OS for other miscellaneous compute and finally Android guest as the host for low criticality processing. For this, we need a method to efficiently deliver messages between the virtual machines, which is the overarching goal of the thesis.


\section{Objectives and Methodology}
The primary objective of the project is to design and implement a software solution to enable cross VM communication. Preliminary research for the project will asses the feasibility of two different methods to achieve this. 


\section{Requirements}
The requirements for the thesis will be primarily set over the course of the research phase, however, there are some primary requirements that are binary in nature. Following is a list of requirements that must be met for the project
\begin{itemize}
	\item Both guest operating systems must have direct read-write access to shared memory across the hypervisor
	\item The latency for communication must be less than 60Hz refresh rate of the infotainment system. This translates to a maximum allowable latency of 33.3 ms for the cross-VM communication.
	\item Android OS kernel module and processing must not pose any viable risk to the high criticality timing guarantees of the RTOS
\end{itemize}

\section{Deliverables}
Following is a list of the tentative deliverable for the thesis. Note that depending on design decisions made over the course of the research, some components may be added or removed, therefore the tentative nature of the list.

\begin{itemize}
	\item Linux or Quest-V (host) driver to set up the shared buffer
	\item Linux or Quest (guest) driver as a producer for buffer
	\item Android (guest) driver as consumer of buffer
	\item Demonstration of the functionality of the above three on hardware (possibly requiring userland programsd)
	\item Thesis
\end{itemize}


\end{document}